% ----------------------------------------------------------------------------------------------------- %
% Manual da Classe UFTeX
% 
% Versão 2.1:   Março 2018
%
% Criado por:   Tiago da Silva Almeida
% Revisado por: Tiago da Silva Almeida
%               Rafael Lima de Carvalho
%               Ary Henrique Morais de Oliveira
%
% https://almeidatiago.github.io/uftex/
% ----------------------------------------------------------------------------------------------------- %

\documentclass[report]{uftex}
% ---- Esse comando cria o nome uftex estilizado
\newcommand\uftex{UF\TeX}

\usepackage{amsmath}
\usepackage{scalefnt}

\usepackage{lipsum}
\usepackage{tikz}
\usepackage[siunitx]{circuitikz}
\usetikzlibrary{arrows}

\usepackage[alf,abnt-emphasize=bf]{abntex2cite}
\renewcommand{\backrefpagesname}{}
\renewcommand{\backref}{}
\renewcommand*{\backrefalt}[4]{}
% ----  Esse comandos são necessário no pré-ambulo para a impressão da lista de lista abreviatuas e de símbolos
\makelosymbols
\makeloabbreviations
% ---- Início do documento
\begin{document}
  % ---- Descrição do título do trabalho 
  \title{1º Seminário Avaliação Financeira de Investimentos}
  % ---- Nome do autor ou autores do trabalho
  \author{Lorenzo}{Costa Miranda}
  % ---- Nome do orientador do trabalho. O último campo representa o título do professor
  % ---- Departamento representa o curso ao qual o trabalho está sendo apresentado. Descrito por meio de duas iniciais do curso
  \department{CE}
  % ---- Data da apresentação do trabalho
  \date{12}{03}{2024}
  % ---- Palavras-chaves em português do trabalho
  \keyword{\LaTeX}
  \keyword{\uftex}
  \keyword{Trabalho de Conclusão de Curso}
  \keyword{Redação Científica}
  \keyword{Extensão Universitária}
  % ---- Palavras-chaves em inglês do trabalho
  \foreignkeyword{\LaTeX}
  \foreignkeyword{\uftex}
  \foreignkeyword{Bachelor Thesis}
  \foreignkeyword{Scientific Writing}
  \foreignkeyword{University Extension}
  % ---- Comando responsável por criar a capa do trabalho e/ou folha de resto conforme a configuração exigida
  \maketitle

  \frontmatter

  \printlosymbols  
  \printloabbreviations
  % ---- Cria a lista de figuras. OPCIONAL
  %\listoffigures
  % ---- Cria a lista de tabelas. OPCIONAL
  %\listoftables 
  % ---- Cria o sumário. OBRIGATÓRIO
 
% --- Marca o inicio dos elementos textuais. Capítulos.
\mainmatter
% ---- Defino o espaçamento de um e meio centímetros
\onehalfspacing
% ----------------------------------------------------------------------------------------------------- %
% Capítulos do trabalho
% ----------------------------------------------------------------------------------------------------- %

\chapter{Capitalização Composta}

\large \textbf{Resolução:}

\section{Taxas de Juros Compostos Equivalentes}


\begin{itemize}
	\item[(a)] $i_q = (1+i_t)^\frac{q}{t} -1$ $\rightarrow$ $i_q = (1+0,33)^\frac{3}{12} -1 = 0,07389$ $\rightarrow$ \textbf{7,38\%}
	
	\item[(b)] $i_q = (1+i_t)^\frac{q}{t} -1$ $\rightarrow$ $i_q = (1+0,025)^\frac{12}{1} -1 = 0,3448$ $\rightarrow$ \textbf{34,48\%}
	
	\item[(c)] $i_q = (1+i_t)^\frac{q}{t} -1$ $\rightarrow$ $i_q = (1+0,04)^\frac{4}{1} -1 = 0,1698$ $\rightarrow$  \textbf{16,98\%}
	
	\item[(d)] $i_q = (1+i_t)^\frac{q}{t} -1$ $\rightarrow$ $i_q = (1+0,06)^\frac{6}{12} -1 = 0,02956$ $\rightarrow$ \textbf{2,95\%}
\end{itemize}

\section{Capitalização Composta}

\begin{itemize}
	\item[(a)] $M = C(1+i)^n$ $\rightarrow$ $M = 6.000(1+0,03)^3 = \textbf{6.556,362}$
	
	\item[(b)] $M = C(1+i)^n$ $\rightarrow$ $M = 125.000(1+0,03)^6 = 149.256,537 - 125.000 = \textbf{24.256,53}$
	
	\item[(c)] $M = C(1+i)^n$ $\rightarrow$ $125.000 = C(1+0,03)^6 = \textbf{104.685,532}$
	
	\item[(d)] $M = C(1+i)^n$ $\rightarrow$ $26.000 = 2.600(1+i)^{28}$ $\rightarrow$ $\frac{26.000}{2600} = (1+i)^{28}$ $\rightarrow$ $10 = (1+i)^{28}$ $\rightarrow$  $\sqrt[28]{10} = 1 + i$ $\rightarrow$ $i =$ \textbf{8,57\%}
	
	\item[(e)] $M = C(1+i)^n$ $\rightarrow$ $2C = C(1+0,03)^n$ $\rightarrow$ $n = \log_{1,03}(2)$ $\rightarrow$ $\frac{\log(2)}{\log(1,03)} = \frac{0,30102}{0,01283}$ $\rightarrow$ \textbf{23,44}
\end{itemize}

\section{Desconto Comercial Composto}

\begin{itemize}
	\item[(a)] 78 dias = 2,5 meses. $D_c = C(1-i)^n$ $\rightarrow$ $1.110,63 = 10.000(1-i)^{2,5}$ $\rightarrow$  $0,111063 = (1-i)^{2,5}$ $\rightarrow$ $\sqrt[2,5]{0,111063} = i-i$ $\rightarrow$ i = \textbf{58,48\%} 
	
	\item[(b)] 51 dias = 1,7 meses. $D_c = C(1-i)^n$ $\rightarrow$ $6.168 = 6.730(1-i)^{1,7}$ $\rightarrow$ $\frac{6.168}{6730} = (1-i)^{1,7}$ $\rightarrow$ $\sqrt[1,7]{0,91649} = 1-i$ $\rightarrow$ i = \textbf{5\%}
	
	\item[(c)] $D_c = C(1-i)^n$ $\rightarrow$ $D_c = 35.000(1-0,05)^3$ $\rightarrow$ $D_c = 35.000 \times 0,857375$ $\rightarrow$ \textbf{30.008,125}
	
\end{itemize}

\section{Valor do Dinheiro no Tempo}

\begin{itemize}
	\item[(a)] $VP = \frac{VR}{(1+i)^n}$ $\rightarrow$ $VP = \frac{190.000}{(1+0,1455)^\frac{152}{360}}$ $\rightarrow$ $VP = \frac{190.000}{1,05903} = \textbf{179.409}$
	 
	\item[(b)] $VP = \frac{VR}{(1+i)^n}$ $\rightarrow$ $VP = \frac{30.000}{(1+0,0117)^\frac{148}{30}}$ $\rightarrow$ $VP = \frac{30.000}{1,059063} = \textbf{28.326}$ 
	
	\item[(c)] $VP = \frac{VR}{(1+i)^n}$ $\rightarrow$ $75.000 = \frac{VR}{(1+0,0113)^\frac{153}{30}}$ $\rightarrow$ $75.000 \times 1,05898 = VR = \textbf{79.423}$
	
	\item[(d)]  $VP = \frac{VR}{(1+i)^n}$ $\rightarrow$ $160.000 = \frac{VR}{(1+0,0892)^\frac{60}{360}}$ $\rightarrow$ $160.000 \times 1,01434 = VR = \textbf{162.294}$ 
\end{itemize}

\section{Séries de Pagamentos}

\begin{itemize}
	\item[(a)] $FAC_{pos}: N = V[\frac{(1+i)^n-1}{i}]$ $\rightarrow$ $N = 700[\frac{(1+0,028)^8-1}{0,028}]$ $\rightarrow$ $N = 700[\frac{0,24722}{0,028}] = \textbf{6.180,5}$ 
	
	\item[(b)] $FAC_{ante}: VT = PMT\frac{(1+i)^n-1}{i}(1+i)$ $\rightarrow$ $VT = 700\frac{(1+0,028)^{10}-1}{0,028}(1+0,028)$ $\rightarrow$ $VT = 700\frac{1,31804 -1}{0,028}(1,028) = 700\frac{0,31804}{0,028}(1-0,028)$ $\rightarrow$ $700\times11,35884\times1,028 = \textbf{8.173,827}$ 
	
	\item[(c)] $FFC_{ante}: = PMT = VF\frac{i}{(1+i)^n-1}\frac{1}{1+i}$ $\rightarrow$ $PMT = 45.000\frac{0,03}{(1+0,03)^{30}-1}\frac{1}{1+0,03}$ $\rightarrow$ $PMT = 45.000\frac{0,03}{1,42726}\frac{1}{0,03}$ $\rightarrow$ $45.000\times0,021019\times0,9708 = \textbf{889,860}$
	
	\item[(d)] ?
	
	\item[(e)] $FAC_{ante}: VF = PMT\frac{(1+i)^n-1}{i}(1+i)$ $\rightarrow$ $VF = 670\frac{(1+0,06)^{12}-1}{0,06}(1+0,06)$ $\rightarrow$ $670\frac{1,012196}{0,06}(1+0,06)$ $\rightarrow$ $670\times16,86994\times1,06 = \textbf{11.981,0322}$
		
	\item[(f)] $122.000 - 40\% = 73.200$ $\rightarrow$ $FRC_{ante}: V = VP\frac{(1+i)^ni}{(1+i)^n-1}\frac{1}{1+i}$ $\rightarrow$ $V = 73.200\frac{(1+0,025)^{24}0,025}{(1+0,025)^{24}-1}\frac{1}{1+0,025}$ $\rightarrow$ $V = 73.200\frac{1,80872\times0,025}{0,80872}\frac{1}{1,025}$ $\rightarrow$ $73.200\times0,055913\times0,97560 = \textbf{3.992,983}$
	
\end{itemize}

\section{Sistema de Amortização PRICE e SAC}

\begin{itemize}
	\item[(a)] PRICE e SAC:
	 $J = I . N \rightarrow 0,02 \times 30.000 = \textbf{600,00}$
	
	\item[(b)] PRICE:
	$R = C[\frac{(1=i)^ni}{(1+i)^n-1}]$ $\rightarrow$ $R = 30.000[\frac{(1+0,02)^{24}0,02}{(1+0,02)^{24}-1} \rightarrow R = 30.000[\frac{0,03216}{0,60843} = 1.586,15$ \\
	\underline{Depois:} $FVA(i,n-t) = [\frac{(1+i)^n-1}{(1+i)^ni}] \rightarrow FVA(0,02, 24-14) FVA = [\frac{0,21899}{0,02437}] = 8,9860$ \\
	 $J_{t} = i . R . FVA (i,n,-1+1) \rightarrow J_{14} = 0,02 \times 1.586,15 \times 8,9860 = \textbf{285,06}$\\
	\underline{Antes:}  $FVA(i,n-t) = [\frac{(1+i)^n-1}{(1+i)^ni}] \rightarrow FVA(0,02, 24-13) FVA = [\frac{0,24337}{0,02486}] = 8,9860$ \\
	$J_{t} = i . R . FVA (i,n,-1+1) \rightarrow J_{14} = 0,02 \times 1.586,15 \times 9,789 = \textbf{310,53}$\\
	
	SAC: 
	Amortização $(A): VP/n \rightarrow 30.000/24 = 1.250$ \\
	\underline{Depois:}
	$J_t = i.A.(n-t+1) \rightarrow J_{14} = 0,02 \times A \times (24-14+1)$ $\rightarrow$ $J_{14} = 0,02 \times 1.250 \times 11 = \textbf{275}$\\
	\underline{Antes:}
	$J_t = i.A.(n-t+1) \rightarrow J_{14} = 0,02 \times A \times (24-13+1)$ $\rightarrow$ $J_{14} = 0,02 \times 1.250 \times 12 = \textbf{300}$ \\

	\item[(c)] PRICE:
	14° parcela da amortização ($A_{14}$): $R - J_{14} \rightarrow A_{14}: 1.586,15 - 285,06 = \textbf{1.301,09}$\\
	SAC: 
	14° parcela da amortização ($A_{14}$): $30.000/24 = \textbf{1.250}$
	 
	\item[(d)] PRICE:
	$Sd_{t} = R.FVA(i,n-t) \rightarrow Sd_{14} = R.FVA(0,02,14) \rightarrow Sd_{14} = 1.586,15 \times 8,9860 = \textbf{14.253,14}$\\
	SAC: 
	$P_{t} = A.(n-t) \rightarrow P_{14} = 1.250 \times 10 = \textbf{12.500}$
	
	\item[(e)] Tabelas:
	
	\begin{table}[h]
		\centering
		\scalefont{1.4}
		\begin{tabular}{c|c|c|c|c}
			\hline
			N° & Prestação & Juros & Amortização & Saldo devedor \\ 
			\hline
			0 & 00 & 00 & 00 & 30.000 \\
			\hline
			1 & 1.586,15 & 600 & 986,15
			 & 29.014,18 \\
			\hline
			2 & 1.586,15 & 580,28 & 1005,87
			 & 28.008,31 \\
			\hline
			3 & 1.586,15 & 560,16 & 1025,99
			 & 26.982,33 \\
			\hline
			4 & 1.586,15 & 539,64 & 1046,51
			 & 25.935,83 \\
			\hline
			5 & 1.586,15 & 518,71 & 1067,44
			 &  24.868,39\\
			\hline
			6 & 1.586,15 & 497,36 & 1088,79
			 & 23.779,61 \\
			\hline
			7 & 1.586,15 & 475,59 & 1110,56
			 & 22.669,05 \\
			\hline
			8 & 1.586,15 & 453,38 & 1132,77
			 & 21.536,28 \\
			\hline
			9 & 1.586,15 & 430,72 & 1155,43
			 & 20.380,86 \\
			\hline
			10 & 1.586,15 & 407,61 & 1178,54
			 & 19.202,33 \\
			\hline
		    11 & 1.586,15 & 384,04 & 1202,11
		     & 18.000,22 \\
			\hline
			12 & 1.586,15 & 360,00 & 1226,15
			 & 16.774,08 \\
			\hline
			13 & 1.586,15 & 335,48 & 1250,67
			 & 15.523,41 \\
			\hline
			14 & 1.586,15 & 310,46 & 1275,69
			 & 14.247,73 \\
			\hline
			15 & 1.586,15 & 284,95 & 1301,2
			 & 12.946,53 \\
			\hline
			16 & 1.586,15 & 258,93 & 1327,22
			 & 11.619,31 \\
			\hline
			17 & 1.586,15 & 232,38 & 1353,77
			 & 10.265,55 \\
			\hline
			18 & 1.586,15 & 205,31 & 1380,84
			 & 8.884,71 \\
			\hline
			19 & 1.586,15 & 177,69 & 1408,46
			 & 7.476,25 \\
			\hline
			20 & 1.586,15 & 149,52 & 1436,63
			 & 6.039,62 \\
			\hline
			21 & 1.586,15 & 120,79 & 1465,36
			 & 4.574,27 \\
			\hline
			22 & 1.586,15 & 91,48 & 1494,67
			 & 3.079,60 \\
			\hline
			23 & 1.586,15 & 61,59 & 1524,56
			 &  1.555,04 \\
			\hline
			23 & 1.586,15 & 31,10 & 1555,05
			 & 0 \\
			\hline
		\end{tabular}
		\caption{Tabela PRICE}
	\end{table}
	
	\begin{table}[h]
	\centering
	\scalefont{1.4}
	\begin{tabular}{c|c|c|c|c}
		\hline
		N° & Prestação & Juros & Amortização & Saldo devedor \\ 
		\hline
		0 & 00 & 00 & 00 & 30.000 \\
		\hline
		1 & 1.850 & 600 & 1.250
		& 28.750 \\
		\hline
		2 & 1.825 & 575 & 1.250
		& 27.500 \\
		\hline
		3 & 1.800 & 550 & 1.250
		& 26.250 \\
		\hline
		4 & 1.775 & 525 & 1.250
		& 25.000 \\
		\hline
		5 & 1.750 & 500 & 1.250
		&  23.750\\
		\hline
		6 & 1.725 & 475 & 1.250
		& 22.500 \\
		\hline
		7 & 1.700 & 450 & 1.250
		& 21.250 \\
		\hline
		8 & 1.675 & 425 & 1.250
		& 20.000 \\
		\hline
		9 & 1.650 & 400 & 1.250
		& 18.750 \\
		\hline
		10 & 1.625 & 375 & 1.250
		& 17.500 \\
		\hline
		11 & 1.600 & 350 & 1.250
		& 16.250 \\
		\hline
		12 & 1.575 & 325 & 1.250
		& 15.000 \\
		\hline
		13 & 1.550 & 300 & 1.250
		& 13.750 \\
		\hline
		14 & 1.525 & 275 & 1.250
		& 12.500 \\
		\hline
		15 & 1.500 & 250 & 1.250
		& 11.250 \\
		\hline
		16 & 1.475 & 225 & 1.250
		& 10.000 \\
		\hline
		17 & 1.450 & 200 & 1.250
		& 8.750 \\
		\hline
		18 & 1.425 & 175 & 1.250
		& 7.500 \\
		\hline
		19 & 1.400 & 150 & 1.250
		& 6.250 \\
		\hline
		20 & 1.375 & 125 & 1.250
		& 5.0002 \\
		\hline
		21 & 1.350 & 100 & 1.250
		& 3.750 \\
		\hline
		22 & 1.325 & 75 & 1.250
		& 2.500 \\
		\hline
		23 & 1.300 & 50 & 1.250
		&  1.250 \\
		\hline
		23 & 1.275 & 25 & 1.250
		& 0 \\
		\hline
	\end{tabular}
	\caption{Tabela SAC}
\end{table}	
	
	
\end{itemize}

\chapter{Formação do Preço de venda e Lucro}

\section{ Com base nos dados da tabela a seguir, calcular o preço de venda da empresa MotorTem Ltda pelo método Mark-up:}

Despesa Variável: $17\% + 1,65\% + 24\% + 1,50\% = 44,15$ \\ 
Despesa Fixa/Lucro: $3\% + 20\% = 30\%$ \\
$44,15\% + 35\% = 74,15\%$  \\
Mark-up multiplicador: $100\% - 74,15\% = 25,85\% \rightarrow \frac{100\%}{25,85\%} = 3,8684$ \\
Preço de Venda: $700 \times 3,8684 = \textbf{2.707,88}$ 

\section{ Calcular o PV para a empresa SeiTudo Ltda para 30 dias pelo método Direto:}

$DA = \frac{(PE \times (1+i)^n - VR) \times i}{(1+i)^n -1}$ $\rightarrow$ $\frac{(95.000(1+0,225)^{10}-30.000) \times 0,225}{(1+0,225)^{10}-1}$ = $\frac{920.790,571 - 30.000}{8,69253}$ = $\frac{890.790,571}{8,69253} = \textbf{102.477,7103}$

\section{A metalúrgica FerroAço Ltda dispõe dos seguintes dados de produção:}

\begin{itemize}
	\item[(a)] $PPV = \frac{125 + 85}{1-0,15} = \frac{210}{0,85} = \textbf{247,05}$ \\
	
$PvP_{30} = PVV \times (1+i)^n$ $\rightarrow$ $247,05(1 + 0,2275)^1 = \textbf{253,843}$ \\

$PvP_{60} = 247,05(1+0,0275)^2$ $\rightarrow$ $1,0557 \times 247,05 = \textbf{266,824}$
	
	\item[(b)] $PE/u = \frac{CF}{MC}$ $\rightarrow$ $\frac{17.000}{85} = \textbf{200 unidades}$ \\
$PE/v = PVV \times PE/v$ $\rightarrow$ $247,05 \times 200 = \textbf{49.410}$ \\
Justificativa pelo DRE: 49.410(vendas) - 17.411 (15\% imposto) - 25.000(Custos diretos). MC = 16.998,5 - 27.000 (Custo fixo) = 00,00
	
	\item[(c)] $MSO/u = 1.300 - 200 = 1.100$ \\
$MSO/v = 1.100 \times 247,05 = 271.755$ \\
$MSO_\% = \frac{MSO/u - vendas(PE/v)}{MSO/u}$ $\rightarrow$ $\frac{1.100-200}{1.100} = \frac{900}{1.100} = 81,81\%$ \\
$\%MC$: $\frac{MC}{PVV}$ $\rightarrow$ $\frac{85}{427,05} = 0,3440$
\text{Lucro}: $\frac{\%MC \times \%MSO}{100\%}$ $\rightarrow$ $\frac{34,40\% \times 31,81}{100\%} = \textbf{28,14\%}$

\end{itemize}

\chapter{Ponto de Equilíbrio Empresarial}

\section{Considerando os dados do item 2.3 da empresa FerroAço Ltda:}

\begin{itemize}
	
\item[(a)] $PEC_q = \frac{CF + DF}{Pu-CVu-DVu} \rightarrow PEC_q = \frac{17.000}{85} = \textbf{200}$ \\

$PEC_v = \frac{CF + DF}{MCu/PVu} \rightarrow PEC_v = \frac{17.000}{85/247,05} \rightarrow PEC_v = \frac{17.000}{0,3440} = \textbf{49.418,60}$\\

$PEE_q = \frac{CF + DF + L}{Pu - CVu - DVu}$ \\
$Lucro: 49.410 \times 0,2814 = 13.903,97$ \\
$PEE_q = \frac{17.000 + 13.903,97}{85} = \textbf{363,57}$ \\

$PEE_v = \frac{CF + DF + L}{MCu/PVu} \rightarrow PEE_v = \frac{17.000 + 13.903,97}{85/247,05} \rightarrow PEE_v = \frac{30.903,97}{0,3440} = \textbf{89.837,12}$

\item [(b)] s

\end{itemize}	

\section{De acordo com a tabela a seguir: Vendas de ferramentas 60\% e motores 40\% das vendas totais. Calcular o PEC e PEE para o mix de vendas adas empresas}

\begin{table}[h]
	\centering
	\begin{tabular}{l|c|c|c}
		\hline
		PRODUTOS & FERRAMENTAS & MOTORES & TOTAL \\ 
		\hline
		Receitas Unitárias & $70,00 \times 0,60 = 42,00$  & $250,00 \times 0,40 = 100$ & 142,00 \\
		\hline
		Custos Variáveis Unitários & $20 \times 0,60 = 12,00$ & $150,00 \times 0,40 = 60,00$ & 72,00 \\
		\textbf{= MC/U} & $50,00 \times 0,60 = 30$ & $100,00 \times 0,40 = 40,00$ & \textbf{70,00} \\
		\hline
	\end{tabular}
\end{table}

\begin{itemize}

\item [PEC:] $PE = (Custo Fixo + Despesa Fixa) /$ Margem de Contribuição \\
$PE = (35.000 / 70) = 500 unidades$ \\
PEC de ferramentas = $500 \times 60\%$ = 300 unidades para ferramentas
PEC de motores = $500 \times 40\%$ = 200 unidades para motores

\item [PEE:] $PE = (Custo Fixo + Despesa Fixa + L) /$ Margem de contribuição \\
$PE = (35.000 + 7.000/ 70) = 600 unidades$ \\
PEE de ferramentas = $600 \times 60\%$ = 360 unidades para ferramentas
PEE de motores = $500 \times 40\%$ = 200 unidades para motores

\end{itemize}

\chapter{Avaliação de Investimento de Produção}	

\section{A empresa MalaTem Ltda opera com capacidade de produção de 100 malas por mês. Atualmente a produção e as vendas são de 80 malas por mês. Os custos de produção são: custos variáveis unitários de R\$ 50,00 e custos fixos unitários de R\$ 30,00. O preço de venda é de R\$ 110,00 a unidade. O fabricante recebe uma encomenda de 10 malas por um valor de R\$ 105,00 a unidade. Deve aceitar essa encomenda?}

Resposta: Sim, deve-se aceitar esse pedido, uma vez que o preço proposto pela encomenda é superior que o custo total por unidade do produto. 

\section{A empresa Tudobom Ltda fabrica bolos na cidade de Palmas. A empresa tem capacidade para fabricar 1.200 unidades mensalmente. A produção do mês está em 80\% de sua capacidade, ou seja, 960 unidades. Seu produto é vendido a R\$ 20,00 por unidade.}

\begin{itemize}

\item [(a)] Tabelas:

\begin{table}[h]
	\centering
	\begin{tabular}{l|l}
		\hline 
		Custo fixo por unidade (R\$1.210,00/1.100 unid) & R\$1,10 \\ 
		\hline
		Custos variáveis por unidade & R\$5,50 \\
		\hline
		Custo total por unidade & R\$6,60 \\
		\hline
	\end{tabular}
\end{table}

\begin{table}[h]
	\centering
	\begin{tabular}{l|l}
		\hline 
		Vendas Líquidas & R\$20,00 \\ 
		\hline
		Custos Variáveis - CPV & (R\$5,50) \\
		\hline
		 Despesas de Vendas Variáveis & R\$00,00 \\
		\hline
		\textbf{= Margem de contribuição} & \textbf{14,50}
	\end{tabular}
\end{table}

É viável aceitar o pedido. 

\item [(b)] Tabelas:

\end{itemize}

\textbf{\begin{table}[h]
		\centering
		\begin{tabular}{l|l}
			\hline 
			Custo fixo por unidade (R\$1.210,00/1.200 unid) & R\$1,0083 \\ 
			\hline
			Custos variáveis por unidade & R\$5,50 \\
			\hline
			Custo total por unidade & R\$6,5083 \\
			\hline
		\end{tabular}
\end{table}}


\begin{table}[h]
	\centering
	\begin{tabular}{l|l}
		\hline 
		Vendas ($100 \times R\$ 20$) & R\$2.000 \\ 
		\hline
		Custos Variáveis ($100 \times R\$ 17,50$) & (R\$1.750) \\
		\hline
		Deixa de ganhar & R\$250,00 \\
		\hline
	\end{tabular}
\end{table}
 

Devemos acrescentar ao custo de produção unitário R\$ 0,73, que é o valor por unidade que a indústria deixará de ganhar com redução das vendas de 100 unidades em Palmas (R\$ 250,00 / 340 unidades = R\$0,73).

\textbf{\begin{table}[h]
		\centering
		\begin{tabular}{l|l}
			\hline 
			Custo fixo por unidade (R\$1.210,00/1.200 unid) & R\$1,0083 \\ 
			\hline
			Custos variáveis por unidade & R\$5,50 + R\$ 0,73 \\
			\hline
			Custo total por unidade & R\$7,2383 \\
			\hline
		\end{tabular}
\end{table}}

\chapter{Decisões de Concessão de Crédito}

\section{Considerando os dados da tabela a seguir da empresa ServeBem Ltda, avaliar as estratégias de concessão de crédito e não concessão de crédito.}

A decisão da empresa de conceder ou não conceder crédito está de acordo com os valores ed valor presente líquido (VPL). Se o VPL de oferecer crédito for maior que o VPL de não conceder, então concluí-se que a concessão trará um retorno maior, o contrário tornaria não vantajoso conceder. 

Não conceder crédito: $VPL = (Po.Qo)-(Co.Qo) \rightarrow VPL = (45,00\times150,00) - (25,00\times150,00) \rightarrow VPL = 6.750 - 3.750 = \textbf{3.000}$ \\

Conceder crédito: $VPL = \frac{h.P'o.Q'o}{1-r_b} - C'o.Q'o \rightarrow VPL = \frac{0,95\times45,00\times250}{1+0,025} - 30,00\times250 \rightarrow VPL = \frac{10.687,5}{1,025} - 7.500 = \textbf{2.926,82}$ \\

Então conclui-se que não é vantajoso conceder crédito uma vez que o $VLP_{semcredtio}$ é maior que $VLP_{comcredito}$.

\section{A empresa Trator Ltda. vende aproximadamente 750 minis tratores por ano, ao preço de R\$ 7.250,00 por unidade. Todas as vendas são efetuadas a prazo, em condições de 3/30 ou 90 dias líquidos. A empresa oferece as condições de pagamentos a seguir: Suponhamos que 65\% dos clientes dessa empresa aproveitem o desconto oferecido pela empresa e paguem no trigésimo dia de cada mês e os demais paguem no nonagésimo dia. Calcule o prazo médio de recebimento (PMR); as vendas diárias médias (VDM) e o saldo de crédito a receber (SCR).}

$PMR = 0,65 \times 30 + 0,35 \times 90$ = \textbf{51 dias.} 

$VDM = \frac{Preço.Quantidade}{365} \rightarrow VDM = \frac{7.250,00 \times 750}{365}$ = \textbf{R\$14.897,26.}

$SCR = PMR.VDM \rightarrow SCR = 51 \times 14.897,26$ = \textbf{R\$759.760,26.}
 
Prova real: $PMR = SCR/VDM = 759.760,26/14.897,26$ = 51.

\chapter{Avaliação de Investimentos em Vendas}

Utilizando-se dos dados operacionais e financeiros do quadro a seguir, elabore um orçamento de vendas dos produtos que a Empresa Atacadista Sobedesse Ltda comercializa: Chuveiro Tradicional (CT) e Chuveiro Jato (CJ). Pede-se: Avaliação do investimento de vendas para outubro e novembro/ 2023 e calcular o lucro líquido e a margem de lucro líquida.

\begin{itemize}
	
\item[(a)] Receitas das vendas dos dois produtos. 

\begin{table}[h]
	\centering
	\begin{tabular}{l|c|c|c}
		\hline
		 & \textbf{CT} & \textbf{CJ} & \textbf{TOTAL} \\ 
		\hline
		\textbf{Volume de vendas em unidades} &  &  &  \\
		\hline
		Agosto 2023 & 10.000 & 5.000 & 15.000 \\
		\hline
		Setembro 2023 & 15.000 & 7.000 & 22.000 \\
		\hline
		Total por produto & 25.000 & 12.000 & 37.000 \\
		\hline
		Aumento das vendas em (\%) & 15 & 10 & \\
		\hline 
		\textbf{Volume de vendas em unidades} & & & \\
		\hline
		Outubro 2023 & 11.500 & 5.500 & 17.000 \\
		\hline
		Novembro 2023 & 17.250 & 7.700 & 24.950 \\ 
		\hline 
		Total por produto & 28.750 & 13.200 & 41.950 \\
		\hline
		Preço de venda em (R\$) & 40 & 60 & \\
		\hline 
		\textbf{Receita de venda em (R\$)} &  & & \\
		\hline 
		Outubro 2023 & 460.000 & 330.000 & 790.000 \\
		\hline 
		Novembro 2023 & 690.000 & 462.000 & 1.152.000 \\
		\textbf{Total por produto} & 1.150.000 & 792.000 & 1.942.000 \\
	\end{tabular}
\end{table}

\item[(b1)] Custos das vendas por produto.

 \begin{table}[h]
 	\centering
 	\begin{tabular}{l|c|c|c}
 		\hline
 		& \textbf{CT} & \textbf{Cj} & \textbf{Total} \\ 
 		\hline
 		\textbf{Outubro 2023} & 232.300 & 166.833 & 399.133 \\
 		\hline
 		Caixas & 2.300 & 1.833,33 & 4.133 \\
 		\hline
 		Chuveiros & 230.000 & 165.000 & 395.000 \\
 		\hline
 		\textbf{Novembro 2023} & 348.450 & 233.566 & 348.450 \\
 		\hline
 		Caixas & 3.450 & 2.566,66 & 6.016 \\
 		\hline
 		Chuveiros & 345.000 & 231.000 & 576.000 \\
 		\hline
 		\textbf{Total por produto} & 580.750 & 400.399 & 981.149 \\
	\end{tabular}
\end{table}


\item[(b2)] Cálculo das compras de chuveiros.

 \begin{table}[h]
	\centering
	\begin{tabular}{l|c|c|c}
		\hline
		& \textbf{CT} & \textbf{Cj} & \textbf{Total} \\ 
		\hline
		Estoque final & 1.000 & 450 & 1.450 \\
		\hline
		Estoque inicial & 2.000 & 750 & 2.750 \\
		\hline
		Quantidade de compras & 27.750 & 12.900 & 40.650 \\
		\hline
		Preço de compras (R\$) & 20 & 30 & \\
		\hline
		\textbf{Total de compras em (R\$)} & 555.000 & 387.000 & 942.000 \\
	\end{tabular}
\end{table}

\vspace*{2cm}

\item[(c)] Cálculo das compras das embalagens: 

 \begin{table}[h]
	\centering
	\begin{tabular}{l|c|c|c}
		\hline
		& \textbf{CT} & \textbf{Cj} & \textbf{Total} \\ 
		\hline
		Estoque final & 50 & 30 & 80 \\
		\hline
		Estoque inicial & 100 & 50 & 150 \\
		\hline
		Quantidade de compras & 28.700 & 13.180 & 40.650 \\
		\hline
		Preço de compras (R\$) & 4 & 5 & \\
		\hline
		\textbf{Total de compras em (R\$)} & 114.800 & 65.900 & 180.700 \\
	\end{tabular}
\end{table}

\item[(d)] Total de compras.

 \begin{table}[h]
	\centering
	\begin{tabular}{l|c|c|c}
		\hline
		& \textbf{CT} & \textbf{Cj} & \textbf{Total} \\ 
		\hline
		Lâmpadas em (R\$) & 555.000 & 387.000 & 942.000 \\
		\hline
		Embalagens em (R\$) & 114.800 & 65.900 & 180.700 \\
		\hline
		\textbf{Total de compras em (R\$)} & 669.800 & 452.900 & 1.122.700 \\
	\end{tabular}
\end{table}

\item[(e)] Lucro bruto em estoque.
\end{itemize}

 \begin{table}[h]
	\centering
	\begin{tabular}{l|c|c|c}
		\hline
		& \textbf{CT} & \textbf{Cj} & \textbf{Total} \\ 
		\hline
		Receita de vendas & 1.150.000 & 792.000 & 1.942.000 \\
		\hline
		(-)Custos das vendas & 669.800 & 452.900 & 1.122.700 \\
		\hline
		\textbf{Total de compras em (R\$)} & 480.200 & 339.100 & 819.300 \\
	\end{tabular}
\end{table}
\chapter{Coeficiente de risco e retorno}

\section{Com as seguintes informações sobre investimentos da empresa Vaivai Ltda nos projetos A e B abaixo. Calcule qual deles propicia melhor compensação entre o risco e retorno.}


\end{document}